\documentclass[12pt,letterpaper,oneside,reqno]{amsart}
\usepackage{amsfonts}
\usepackage{amsmath}
\usepackage{amssymb}
\usepackage{amsthm}
\usepackage{float}
\usepackage{mathrsfs}
\usepackage{colonequals}
\usepackage[font=small,labelfont=bf]{caption}
\usepackage[left=1in,right=1in,bottom=1in,top=1in]{geometry}
\usepackage[pdfpagelabels,hyperindex,colorlinks=true,linkcolor=blue,urlcolor=magenta,citecolor=green]{hyperref}
\usepackage{graphicx}
\linespread{1.7}
\emergencystretch=1em
\usepackage{array}
\usepackage{etoolbox}
\apptocmd{\sloppy}{\hbadness 10000\relax}{}{}
\raggedbottom

\newcommand \anglePower [2]{\langle #1 \rangle \sp{#2}}
\newcommand \bernoulli [2][B] {{#1}\sb{#2}}
\newcommand \curvePower [2]{\{#1\}\sp{#2}}
\newcommand \coeffA [3][A] {{\mathbf{#1}} \sb{#2,#3}}
\newcommand \polynomialP [4][P]{{\mathbf{#1}}\sp{#2} \sb{#3}(#4)}

% ordinary derivatives
\newcommand \derivative [2] {\frac{d}{d #2} #1}                              % 1 - function; 2 - variable;
\newcommand \pderivative [2] {\frac{\partial #1}{\partial #2}}               % 1 - function; 2 - variable;
\newcommand \qderivative [1] {D_{q} #1}                                      % 1 - function
\newcommand \nqderivative [1] {D_{n,q} #1}                                   % 1 - function
\newcommand \qpowerDerivative [1] {\mathcal{D}_q #1}                         % 1 - function;
\newcommand \finiteDifference [1] {\Delta #1}                                % 1 - function;
\newcommand \pTsDerivative [2] {\frac{\partial #1}{\Delta #2}}               % 1 - function; 2 - variable;

% high order derivatives
\newcommand \derivativeHO [3] {\frac{d^{#3}}{d {#2}^{#3}} #1}                % 1 - function; 2 - variable; 3 - order
\newcommand \pderivativeHO [3]{\frac{\partial^{#3}}{\partial {#2}^{#3}} #1}
\newcommand \qderivativeHO [2] {D_{q}^{#2} #1}                               % 1 - function; 2 - order
\newcommand \qpowerDerivativeHO [2] {\mathcal{D}_{q}^{#2} #1}                % 1 - function; 2 - order
\newcommand \finiteDifferenceHO [2] {\Delta^{#2} #1}                         % 1 - function; 2 - order
\newcommand \pTsDerivativeHO [3] {\frac{\partial^{#3}}{\Delta {#2}^{#3}} #1} % 1 - function; 2 - variable;

% central factorials and related symbols
\newcommand \centralFactorial [2] {#1^{[#2]}}
\newcommand \fallingFactorial [2] {\left(#1 \right)^{\underline{#2}}}
\newcommand{\stirlingii}{\genfrac{\{}{\}}{0pt}{}}
\newcommand{\eulerianNumber}{\genfrac{\langle}{\rangle}{0pt}{}}

% for llceil coeffcient
\newcommand{\nobarfrac}{\genfrac{}{}{0pt}{}}
\def\llceil{\left\lceil\kern-3.5pt\left\lceil}
\def\rrfloor{\right\rfloor\kern-3.5pt\right\rfloor}
\newcommand \llceilCoefficient [3] {\llceil \nobarfrac{#1}{#2} \rrfloor_{#3}}


\newtheorem{thm}{Theorem}[section]
\newtheorem{cor}[thm]{Corollary}
\newtheorem{lem}[thm]{Lemma}
\newtheorem{examp}[thm]{Example}
\newtheorem{conj}[thm]{Conjecture}
\newtheorem{defn}[thm]{Definition}

\numberwithin{equation}{section}

\title[LaTeX Template for Github]
{LaTeX Template for Github}
\author[Petro Kolosov]{Petro Kolosov}
\address{Software Developer, DevOps Engineer}
\email{kolosovp94@gmail.com}
\urladdr{https://kolosovpetro.github.io}
\keywords{
    Keyword1, Keyword2
}
\subjclass[2010]{26E70, 05A30}
\date{\today}
\hypersetup{
    pdftitle={LaTeX Template for Github},
    pdfsubject={
        Your Subject List
    },
    pdfauthor={Petro Kolosov},
    pdfkeywords={
        Your Keywords list
    }
}
\begin{document}
    \begin{abstract}
        The polynomial $\mathbf{P}^m_b(x)$ is a $2m+1$ degree polynomial in $(x,b) \in \mathbb{R}$
defined by a polynomial identity for odd-powers such that derived applying certain interpolation approaches.
This manuscript provides a comprehensive historical survey of the milestones and evolution of the polynomial
$\mathbf{P}^m_b(x)$, continuing with related works based on it.
%In particular, the polynomial $\mathbf{P}^m_b(x)$ establishes a relation between
%ordinary and partial derivatives of odd-powers.
%Also, it gives an opportunity to express odd-power identity in terms of multiplication of certain matrices.

%such that derived applying certain methods of interpolation, so that initially we reach the base case for $m=1$
%generalizing it up to $m\in\mathbb{N}$ afterward.
%In particular, the polynomial $\mathbf{P}^m_b(x)$ may be successfully applicable
%for polynomial interpolation and approximation approaches.
%This manuscript provides a comprehensive historical survey of the milestones and evolution of $\mathbf{P}^m_b(x)$
%as well as related works such that based onto, for instance various polynomial identities, differential equations etc.
%In addition, future research directions are proposed and discussed.

    \end{abstract}

    \maketitle

    \tableofcontents


    \section{Introduction} \label{sec:introduction}
    For the context's sake, we start this manuscript considering the definition and examples of polynomials $\polynomialP{m}{b}{x}$.
Polynomials $\polynomialP{m}{b}{x}$ are of $2m+1$ order in $x,b \in \mathbb{R}$ and defined as follows
\begin{equation}
    \polynomialP{m}{b}{x} = \sum_{k=0}^{b-1} \sum_{r=0}^{m} \coeffA{m}{r} k^r(x-k)^r
    \label{eq:definition_polynomial_p}
\end{equation}
where $\coeffA{m}{r}$ are real coefficients.
To provide more clarity, consider few examples of polynomials $\polynomialP{m}{b}{x}$ for $m=0,1,2$
\begin{equation*}
    \begin{split}
        \polynomialP{0}{b}{x}
        &=b, \\
        \polynomialP{1}{b}{x}
        &=3 b^2 - 2 b^3 - 3 b x + 3 b^2 x, \\
        \polynomialP{2}{b}{x}
        &=10 b^3 - 15 b^4 + 6 b^5 \\
        &- 15 b^2 x + 30 b^3 x - 15 b^4 x \\
        &+ 5 b x^2 - 15 b^2 x^2 + 10 b^3 x^2, \\
        \polynomialP{3}{b}{x}
        &=-7 b^2 + 28 b^3 - 70 b^5 + 70 b^6 - 20 b^7 \\
        &+ 7 b x - 42 b^2 x + 175 b^4 x - 210 b^5 x + 70 b^6 x \\
        &+ 14 b x^2 - 140 b^3 x^2 + 210 b^4 x^2 - 84 b^5 x^2 \\
        &+ 35 b^2 x^3 - 70 b^3 x^3 + 35 b^4 x^3
    \end{split}
\end{equation*}
Comprehensive review and properties of polynomials $\polynomialP{m}{b}{x}$ is given at~\cite{kolosov2016link}.



    \section{Conclusions}\label{sec:conclusions}
    Conclusions of your manuscript.

    \bibliographystyle{unsrt}
    \bibliography{GithubLatexTemplate}
    \noindent \textbf{Version:} \texttt{Local-0.1.0}

\end{document}
