\begin{itemize}
    \item Differential equation~\eqref{eq:odd-exponential-identity} can also be expressed in terms of backward
    and central differential operators, including derivatives on time-scales so that results of~\cite{kolosov2016study}
    could be generalized further.
    \item Theorem~\eqref{eq:odd-power-theorem} provides an opportunity to express odd-power identity
    in terms of multiplication of certain matrices.
    \item There are Taylor series and Maclaurin series versions in terms of $\polynomialP{m}{b}{x}$.
    \item The summation bounds of definition~\eqref{eq:definition_polynomial_p} can be altered so that
    $k$ runs over $1 \leq k \leq b$, by symmetry.
    \item Prove that $\polynomialP{m}{b}{x}$ is an integer valued polynomial in $(x,b)$.
    \item The definition~\eqref{eq:definition_polynomial_p} is closely related to discrete convolution because
    \begin{equation*}
        \polynomialP{m}{b}{x} = \sum_{r=0}^{m} \coeffA{m}{r} \sum_{k=0}^{b-1} k^r(x-k)^r
    \end{equation*}
    where $\sum_{k=0}^{b-1} k^r(x-k)^r$ is the discrete convolution of $x^r$.
    It is worth to get a closer look into it so that new relations in terms of discrete convolution may be found.
    \item All kinds of derivatives e.g.\ forward, backward and central, including the derivatives on time-scales can be expressed
    as double limit of $\polynomialP{m}{b}{x}$ extending the results of~\cite{kolosov_2024_10575485}.
    \item Introducing the definitions of the coefficients
    $\brackCoefficient{n}{k}{m}$ and $\braceCoefficient{n}{k}{m}$
    \begin{align*}
        \brackCoefficient{n}{k}{m} &= \sum_{r=0}^{m} \coeffA{m}{r} k^r (n-k)^r \\
        \braceCoefficient{n}{r}{m} &= \sum_{k=0}^{n-1} \coeffA{m}{r} k^r (n-k)^r
    \end{align*}
    the novel identities can be reached, for example
    \begin{align*}
        \brackCoefficient{2t+1}{1}{m} &= \brackCoefficient{t+2}{2}{m} \\
        \brackCoefficient{n}{k}{m} &= \brackCoefficient{n}{n-k}{m} \\
        \brackCoefficient{2t-3r}{r}{m} &= \brackCoefficient{t}{2r}{m} = \brackCoefficient{2t-3r}{2t-4r}{m}
    \end{align*}
    so that combinatorial sense of above is also a topic to research.
    \item Contribute new OEIS sequences related to $\brackCoefficient{n}{k}{m}$ and $\braceCoefficient{n}{k}{m}$.
    \item An identity
    \begin{align*}
    (x-2a)
        ^{2m+1} = \sum_{r=0}^{m} \coeffA{m}{r} \sum_{k=a+1}^{x-a} (k-a)^r (x-k-a)^r
    \end{align*}
    allows to provide a novel proof of power rule in terms of derivatives of polynomials.
    \item Following the results of~\url{https://arxiv.org/pdf/1603.02468v15.pdf},
    the equation~\eqref{eq:definition_polynomial_p} approximates the odd-power polynomial $x^{2m+1}$ around given points
    $x_i$ as it may be observed from the following plots
    \begin{figure}[H]
        \centering
        \includegraphics[width=1\textwidth]{images/n^3_approximation_m1_b3}
        ~\caption{Approximation of $x^3$.}\label{fig:approximation-n3}
    \end{figure}
    \begin{figure}[H]
        \centering
        \includegraphics[width=1\textwidth]{images/n^5_approximation_m2_b3}
        ~\caption{Approximation of $x^5$.}\label{fig:approximation-n5}
    \end{figure}
    \item English grammar reviews and improvements are welcome.
    \item Improvements and suggestions to current manuscript under open-source initiatives at
    \url{https://github.com/kolosovpetro/HistoryAndOverviewOfPolynomialP}
\end{itemize}
