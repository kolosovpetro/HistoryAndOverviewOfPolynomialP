Back then, in 2016 being a student at the faculty of mechanical engineering,
I remember myself playing with finite differences of the polynomial $n^3$ over the domain of natural numbers $n\in\mathbb{N}$
having at most $0 \leq n \leq 20$ values.
Looking to the values in my finite difference tables, the first and very naive question that came to my mind was
\begin{center}
    \textit{Is it possible to re-assemble the value of
    the polynomial $n^3$ backwards having its finite differences?}
\end{center}
The answer to this question is definitely \textit{Yes}, utilizing the interpolation principles.
Interpolation is a process of finding new data points based on the range of a discrete set of known data points.
Interpolation has been well-developed in between 1674--1684
by Issac Newton's fundamental works, nowadays known as foundation of classical interpolation
theory~\cite{meijering2002chronology}.

At that time, in 2016, I was a first-year mechanical engineering undergraduate,
so that due to lack of knowledge and perspective of view I started re-inventing interpolation
formula myself, fueled by purest passion and feeling of mystery.
All mathematical laws and relations exist from the very beginning, but we only find and describe them, I thought.
That mindset truly inspired me so that my own mathematical journey has been started.
Let us begin considering the table of finite differences of the polynomial $n^3$
\begin{table}[H]
    \begin{center}
        \setlength\extrarowheight{-6pt}
        \begin{tabular}{c|cccc}
            $n$ & $n^3$ & $\Delta(n^3)$ & $\Delta^2(n^3)$ & $\Delta^3(n^3)$ \\
            \hline
            0   & 0     & 1             & 6               & 6               \\
            1   & 1     & 7             & 12              & 6               \\
            2   & 8     & 19            & 18              & 6               \\
            3   & 27    & 37            & 24              & 6               \\
            4   & 64    & 61            & 30              & 6               \\
            5   & 125   & 91            & 36              &                 \\
            6   & 216   & 127           &                 &                 \\
            7   & 343   &               &                 &
        \end{tabular}
    \end{center}
    \caption{Table of finite differences of the polynomial $n^3$.} \label{tab:table}
\end{table}
First and foremost, we can observe that finite difference $\Delta(n^3)$ of the polynomial $n^3$
can be expressed through summation over $n$, e.g
\begin{align}
    \label{eq:cubes_interpolation}
    \begin{split}
        \Delta(0^3) &= 1+6 \cdot 0 \\
        \Delta(1^3) &= 1+6\cdot0+6\cdot1 \\
        \Delta(2^3) &= 1+6\cdot0+6\cdot1+6\cdot2 \\
        \Delta(3^3) &= 1+6\cdot0+6\cdot1+6\cdot2+6\cdot3 \\
        &\; \; \vdots
    \end{split}
\end{align}
Finally reaching its generic form
\begin{equation}
    \Delta(n^3) = 1+6\cdot0+6\cdot1+6\cdot2+6\cdot3+\cdots+6\cdot n = 1 + 6 \sum_{k=0}^{n} k\label{eq:general-cube-eq}
\end{equation}
The one experienced mathematician would immediately notice a spot to apply Faulhaber's formula~\cite{beardon1996sums}
to expand the term $\sum_{k=0}^{n} k$ reaching expected result that matches Binomial theorem~\cite{abramowitz1988handbook},
so that
\begin{equation*}
    \sum_{k=0}^{n} k = \frac{1}{2}(n+n^2)
\end{equation*}
Then our relation~\eqref{eq:general-cube-eq} immediately turns into Binomial expansion
\begin{equation}
    \Delta(n^3) = (n+1)^3 - n^3 = 1 + 6 \left[ \frac{1}{2}(n+n^2) \right] = 1 + 3 n + 3 n^2 = \sum_{k=0}^{2} \binom{3}{k} n^k
    \label{eq:cubes-difference-binomial-theorem}
\end{equation}
However, as was said, I was not the experienced one mathematician back then,
so that I reviewed the relation~\eqref{eq:general-cube-eq} from a little bit different perspective.
Not following the convenient solution~\eqref{eq:cubes-difference-binomial-theorem},
I have rearranged the first order finite differences from the table~\eqref{tab:table} using~\eqref{eq:cubes_interpolation}
to get the polynomial $n^3$
\begin{align}
    \label{eq:rearrangement_to_get_cubes}
    n^3 &= [1+6\cdot0]+[1+6\cdot0+6\cdot1]+[1+6\cdot0+6\cdot1+6\cdot2]+\cdots \nonumber \\
    &+[1+6\cdot0+6\cdot1+6\cdot2+\cdots+6\cdot(n-1)]
\end{align}
Then, rearranging the terms of the equation~\eqref{eq:rearrangement_to_get_cubes} so that it turns into summation
in terms of $k (n-k)$
\begin{equation*}
    \begin{split}
        n^3 &= n + [(n-0) \cdot 6 \cdot 0] + [(n-1)\cdot6\cdot1] + [(n-2)\cdot6\cdot2] + \cdots \\
        &\cdots + [(n-k)\cdot 6 \cdot k] + \cdots + [1\cdot6\cdot(n-1)]
    \end{split}
\end{equation*}
Gives the interpolation of the polynomial $n^3$
\begin{equation}
    \label{eq:cube_identity}
    n^3 = \sum_{k=1}^{n} 6k(n-k) + 1
\end{equation}
It is immediately seen that~\eqref{eq:cube_identity} is true by observing the table of $6k(n-k) + 1$ values
\begin{table}[H]
    \setlength\extrarowheight{-6pt}
    \begin{tabular}{c|cccccccc}
        $n/k$ & 0 & 1  & 2  & 3  & 4  & 5  & 6  & 7 \\
        \hline
        0     & 1 &    &    &    &    &    &    &   \\
        1     & 1 & 1  &    &    &    &    &    &   \\
        2     & 1 & 7  & 1  &    &    &    &    &   \\
        3     & 1 & 13 & 13 & 1  &    &    &    &   \\
        4     & 1 & 19 & 25 & 19 & 1  &    &    &   \\
        5     & 1 & 25 & 37 & 37 & 25 & 1  &    &   \\
        6     & 1 & 31 & 49 & 55 & 49 & 31 & 1  &   \\
        7     & 1 & 37 & 61 & 73 & 73 & 61 & 37 & 1
    \end{tabular}
    \caption{Values of $6k(n-k) + 1$.
    See the OEIS entry: \href{https://oeis.org/A287326}{\texttt{A287326}}~\cite{kolosov2017third}.
    Sequences such that row sums give the polynomials $n^{5}$ and $n^7$
        are also registered in OEIS~\cite{kolosov2018fifth, kolosov2018seventh}.}
    \label{tab:fig_1}
\end{table}
Therefore, we have reached our base case by successfully interpolating the polynomial $n^3$.
Fairly enough that the next curiosity would be
\begin{center}
    \textit{Well, if the relation~\eqref{eq:cube_identity} true for the polynomial $n^3$,
        then is it true that~\eqref{eq:cube_identity} can be generalized for higher powers, e.g. for $n^4$ or $n^5$ either?}
\end{center}
That was my next question, however without any expectation of the final form of generalized relation.
Soon enough my idea was caught by other people.
In 2018, Albert Tkaczyk has published two of his works~\cite{tkaczyk2018problem, tkaczyk2018continuation}
showing the cases for polynomials $n^5, \; n^7$ and $n^9$ that were obtained similarly as~\eqref{eq:cube_identity}.
In short, it appears that relation~\eqref{eq:cube_identity} could be generalized
for any non-negative odd power $2m+1$ solving a system of linear equations.
It was proposed that the case for $n^5$ has explicit form
\begin{equation*}
    n^5 = \sum_{k=1}^{n} \left[ A k^2(n-k)^2 + Bk(n-k) + C \right]
\end{equation*}
where $A,B,C$ are yet-unknown coefficients.
Denote $A,B,C$ as $\coeffA{2}{0}, \coeffA{2}{1}, \coeffA{2}{2}$
to reach the form of a compact double sum
\begin{equation*}
    n^5 = \sum_{k=1}^{n} \sum_{r=0}^{2} \coeffA{2}{r} k^r (n-k)^r
\end{equation*}
Observing the equation above, the potential form of generalized odd-power identity becomes more obvious.
To evaluate the coefficients $\coeffA{2}{0}, \coeffA{2}{1}, \coeffA{2}{2}$
it is necessary construct and solve a system of linear equations following the process
\begin{equation*}
    \begin{split}
        n^5 &= \sum_{r=0}^{2} \coeffA{2}{r} \sum_{k=1}^{n} k^r (n-k)^r \\
        &= \coeffA{2}{0} \sum_{k=1}^{n} k^0 (n-k)^0 + \coeffA{2}{1} \sum_{k=1}^{n} k^1 (n-k)^1 + \coeffA{2}{2} \sum_{k=1}^{n} k^2 (n-k)^2
    \end{split}
\end{equation*}
Expand the terms $\sum_{k=1}^{n} k^r (n-k)^r$ applying the
Faulhaber's formula~\cite{beardon1996sums}
to get the equation
\begin{equation*}
    \coeffA{m}{0} n
    + \coeffA{m}{1} \left[ \frac{1}{6} (-n + n^3) \right]
    + \coeffA{m}{2} \left[ \frac{1}{30} (-n + n^5) \right] - n^5 = 0
\end{equation*}
Multiplying by $30$ both right-hand side and left-hand side, we get
\begin{equation*}
    30 \coeffA{2}{0} n + 5 \coeffA{2}{1} (-n + n^3) + \coeffA{2}{2} (-n + n^5) - 30n^5 = 0
\end{equation*}
Expanding the brackets and rearranging the terms gives
\begin{equation*}
    30 \coeffA{2}{0} - 5 \coeffA{2}{1} n + 5 \coeffA{2}{1} n^3 - \coeffA{2}{2} n + \coeffA{2}{2} n^5 - 30n^5 = 0
\end{equation*}
Combining the common terms yields
\begin{equation*}
    n (30 \coeffA{2}{0} - 5 \coeffA{2}{1} - \coeffA{2}{2}) + 5 \coeffA{2}{1} n^3 + n^5 (\coeffA{2}{2} - 30) = 0
\end{equation*}
Therefore, the system of linear equations follows
\begin{equation*}
    \begin{cases}
        30 \coeffA{2}{0} - 5 \coeffA{2}{1} - \coeffA{2}{2} = 0 \\
        \coeffA{2}{1} = 0 \\
        \coeffA{2}{2} - 30 = 0
    \end{cases}
\end{equation*}
Solving it, we get
\begin{equation*}
    \begin{cases}
        \coeffA{2}{2} = 30 \\
        \coeffA{2}{1} = 0 \\
        \coeffA{2}{0} = 1
    \end{cases}
\end{equation*}
So that the odd-power identity holds
\begin{equation*}
    n^5 = \sum_{k=1}^{n} 30k^2(n-k)^2 + 1
\end{equation*}
It is also clearly seen
why the above identity is true by arranging the terms $30k^2(n-k)^2 + 1$ over $0 \leq k \leq n$ as tabular.
See the OEIS sequence~\cite{kolosov2018fifth}
\begin{table}[H]
    \setlength\extrarowheight{-6pt}
    \begin{tabular}{c|cccccccc}
        $n/k$ & 0 & 1    & 2    & 3    & 4    & 5    & 6    & 7 \\
        \hline
        0     & 1 &      &      &      &      &      &      &   \\
        1     & 1 & 1    &      &      &      &      &      &   \\
        2     & 1 & 31   & 1    &      &      &      &      &   \\
        3     & 1 & 121  & 121  & 1    &      &      &      &   \\
        4     & 1 & 271  & 481  & 271  & 1    &      &      &   \\
        5     & 1 & 481  & 1081 & 1081 & 481  & 1    &      &   \\
        6     & 1 & 751  & 1921 & 2431 & 1921 & 751  & 1    &   \\
        7     & 1 & 1081 & 3001 & 4321 & 4321 & 3001 & 1081 & 1
    \end{tabular}
    \caption{Values of $30k^2(n-k)^2 + 1$.
    See the OEIS entry \href{https://oeis.org/A300656}{\texttt{A300656}}}
    \label{tab:row-sums-gives-fifth-power}
\end{table}

Now we can see that the relation~\eqref{eq:cube_identity}
we got via interpolation of cubes
can be generalized for all non-negative odd-powers $2m+1$ by constructing
and solving a certain system of linear equations.
Therefore, the generalized form of odd-power identity has the form
\begin{equation}
    n^{2m+1} = \sum_{r=0}^{m} \coeffA{m}{r} \sum_{k=1}^{n} k^{r} (n-k)^{r}\label{eq:odd-power-identity}
\end{equation}
where $\coeffA{m}{r}$ are real coefficients.
In more details, the identity~\eqref{eq:odd-power-identity} is discussed
separately in~\cite{kolosov2022106, kolosov2023polynomial}.

However, constructing and solving a system of linear equations for every odd-power $2m+1$ requires a huge effort,
there must be a formula that generates a set of real coefficients $\coeffA{m}{r}$ for each fixed $m$, I thought.
As it turned out, that assumption was correct.
So that I reached MathOverflow community in search of answers that arrived quite shortly.
In~\cite{alekseyev2018mathoverflow}, Dr. Max Alekseyev has provided a complete and comprehensive formula to calculate
coefficient $\coeffA{m}{r}$ for each natural $m\geq 0, \; 0 \leq r \leq m$.
The main idea of Alekseyev's approach was to utilize dynamic programming methods to evaluate the $\coeffA{m}{r}$ recursively,
taking base case $\coeffA{m}{m}$ then evaluating the next coefficient $\coeffA{m}{m-1}$ via backtracking,
continuing similarly up to $\coeffA{m}{0}$.
Before we consider the derivation of the recurrent formula for coefficients $\coeffA{m}{r}$,
a few words must be said regarding the Faulhaber's formula~\cite{beardon1996sums}
\begin{equation*}
    \sum_{k=1}^{n} k^{p} = \frac{1}{p+1}\sum_{j=0}^{p} \binom{p+1}{j} \bernoulli{j} n^{p+1-j}
\end{equation*}
it is important to notice that iteration step $j$ is bounded by the value of exponent $p$,
while the upper bound of the binomial coefficient is $p+1$.
That means we cannot omit summation bounds letting $j$ run over infinity,
unless we perform the following action on the Faulhaber's formula
\begin{equation}
    \begin{split}
        \sum_{k=1}^{n} k^{p}
        = \frac{1}{p+1}\sum_{j=0}^{p} \binom{p+1}{j} \bernoulli{j} n^{p+1-j}
        &= \left[ \frac{1}{p+1}\sum_{j=0}^{p+1} \binom{p+1}{j} \bernoulli{j} n^{p+1-j} \right] - \bernoulli{p+1} \\
        &= \left[ \frac{1}{p+1}\sum_{j} \binom{p+1}{j} \bernoulli{j} n^{p+1-j} \right] - \bernoulli{p+1}
    \end{split}\label{eq:faulhabers-formula}
\end{equation}
At this point we are good to go through the entire derivation of the recurrent formula for
coefficients $\coeffA{m}{r}$.
Applying both Binomial theorem and Faulhaber's formula~\eqref{eq:faulhabers-formula}
to the equation~\eqref{eq:odd-power-identity} we get
\begin{equation*}
    \begin{split}
        &\sum_{k=1}^{n} k^{r} (n-k)^{r}
        = \sum_{t=0}^{r} (-1)^t \binom{r}{t} n^{r-t} \sum_{k=1}^{n} k^{t+r} \\
        &= \sum_{t=0}^{r} (-1)^t \binom{r}{t} n^{r-t} \left[ \frac{1}{t+r+1} \sum_{j} \binom{t+r+1}{j} \bernoulli{j} n^{t+r+1-j} - \bernoulli{t+r+1} \right] \\
        &= \sum_{t=0}^{r} \binom{r}{t} \left[ \frac{(-1)^t}{t+r+1} \sum_{j} \binom{t+r+1}{j} \bernoulli{j} n^{2r+1-j} - \bernoulli{t+r+1} n^{r-t} \right] \\
        &= \sum_{t=0}^{r} \binom{r}{t} \frac{(-1)^t}{t+r+1} \sum_{j} \binom{t+r+1}{j} \bernoulli{j} n^{2r+1-j} - \sum_{t=0}^{r} \binom{r}{t} \frac{(-1)^t}{t+r+1} \bernoulli{t+r+1} n^{r-t} \\
        &= \sum_{j} \sum_{t} \binom{r}{t} \frac{(-1)^t}{t+r+1} \binom{t+r+1}{j} \bernoulli{j} n^{2r+1-j} - \sum_{t=0}^{r} \binom{r}{t} \frac{(-1)^t}{t+r+1} \bernoulli{t+r+1} n^{r-t} \\
        &= \sum_{j} \bernoulli{j} n^{2r+1-j} \sum_{t} \binom{r}{t} \frac{(-1)^t}{t+r+1} \binom{t+r+1}{j} - \sum_{t=0}^{r} \binom{r}{t} \frac{(-1)^t}{t+r+1} \bernoulli{t+r+1} n^{r-t}
    \end{split}
\end{equation*}
We can notice that
\begin{equation}
    \sum_{t} \binom{r}{t} \frac{(-1)^t}{r+t+1} \binom{r+t+1}{j}
    =\begin{cases}
         \frac{1}{(2r+1) \binom{2r}r} & \text{if } j=0\\
         \frac{(-1)^r}{j} \binom{r}{2r-j+1} & \text{if } j>0
    \end{cases}\label{eq:combinatorial-identity}
\end{equation}
An elegant proof of the binomial identity~\eqref{eq:combinatorial-identity} is presented in~\cite{scheuer2023mathstackexchange}.

In particular, the equation~\eqref{eq:combinatorial-identity} is zero for $0< t \leq j$.
So that taking $j=0$ we have
\begin{equation*}
    \begin{split}
        \sum_{k=1}^{n} k^{r} (n-k)^{r}
        &= \frac{1}{(2r+1) \binom{2r}r} n^{2r+1} + \left[ \sum_{j \geq 1} \bernoulli{j} n^{2r+1-j} \sum_{t} \binom{r}{t} \frac{(-1)^t}{t+r+1} \binom{t+r+1}{j} \right] \\
        &- \left[ \sum_{t=0}^{r} \binom{r}{t} \frac{(-1)^t}{t+r+1} \bernoulli{t+r+1} n^{r-t} \right]
    \end{split}
\end{equation*}
Simplifying above equation via~\eqref{eq:combinatorial-identity} we get
\begin{equation*}
    \begin{split}
        \sum_{k=1}^{n} k^{r} (n-k)^{r}
        &= \frac{1}{(2r+1) \binom{2r}r} n^{2r+1}
        + \underbrace{\left[ \sum_{j \geq 1} \frac{(-1)^r}{j} \binom{r}{2r-j+1} \bernoulli{j} n^{2r-j+1} \right]}_{(\star)} \\
        &- \underbrace{\left[ \sum_{t=0}^{r} \binom{r}{t} \frac{(-1)^t}{t+r+1} \bernoulli{t+r+1} n^{r-t} \right]}_{(\diamond)}
    \end{split}
\end{equation*}
Hence, introducing $\ell=2r-j+1$ to $(\star)$ and $\ell=r-t$ to $(\diamond)$
we collapse the common terms in the equation above so that
\begin{equation*}
    \begin{split}
        \sum_{k=1}^{n} k^{r} (n-k)^{r}
        &= \frac{1}{(2r+1) \binom{2r}r} n^{2r+1}
        + \left[ \sum_{\ell} \frac{(-1)^r}{2r+1-\ell} \binom{r}{\ell} \bernoulli{2r+1-\ell} n^{\ell} \right] \\
        &- \left[ \sum_{\ell} \binom{r}{\ell} \frac{(-1)^{r-\ell}}{2r+1-\ell} \bernoulli{2r+1-\ell} n^{\ell} \right]\\
        &= \frac{1}{(2r+1) \binom{2r}r} n^{2r+1} + 2 \sum_{\mathrm{odd \; \ell}} \frac{(-1)^r}{2r+1-\ell} \binom{r}{\ell} \bernoulli{2r+1-\ell} n^{\ell}
    \end{split}
\end{equation*}
Assuming that $\coeffA{m}{r}$ is defined by~\eqref{eq:odd-power-identity},
we obtain the following relation for polynomials in $n$
\begin{equation*}
    \sum_{r} \coeffA{m}{r} \frac{1}{(2r+1) \binom{2r}r} n^{2r+1}
    + 2 \sum_{r} \coeffA{m}{r} \sum_{\mathrm{odd \; \ell}} \frac{(-1)^r}{2r+1-\ell} \binom{r}{\ell} \bernoulli{2r+1-\ell} n^{\ell}
    \equiv n^{2m+1}
\end{equation*}
Replacing odd $\ell$ by $d$ we get
\begin{equation}
    \begin{split}
        &\sum_{r} \coeffA{m}{r} \frac{1}{(2r+1) \binom{2r}r} n^{2r+1}
        + 2 \sum_{r} \coeffA{m}{r} \sum_{d} \frac{(-1)^r}{2r-2d} \binom{r}{2d+1} \bernoulli{2r-2d} n^{2d+1}
        \equiv n^{2m+1} \\
        &\sum_{r} \coeffA{m}{r} \left[ \frac{1}{(2r+1) \binom{2r}r} n^{2r+1} \right]
        + 2 \sum_{r} \coeffA{m}{r} \left[ \sum_{d} \frac{(-1)^r}{2r-2d} \binom{r}{2d+1} \bernoulli{2r-2d} n^{2d+1} \right]
        - n^{2m+1} = 0
    \end{split}\label{eq:equation7}
\end{equation}
Let be $r=m$, then taking the coefficient of $n^{2m+1}$ in~\eqref{eq:equation7} we get
\begin{equation*}
    \coeffA{m}{m} = (2m+1)\binom{2m}{m}
\end{equation*}
and taking the coefficient of $n^{2d+1}$ for an integer $d$ in the range $m/2 \leq d < m$, we get
\begin{equation*}
    \coeffA{m}{d} = 0
\end{equation*}
Taking the coefficient of $n^{2d+1}$ for $d$ in the range $m/4 \leq d < m/2$ we get
\begin{equation*}
    \coeffA{m}{d} \frac{1}{(2d+1) \binom{2d}{d}}
    +2 (2m+1) \binom{2m}{m}\binom{m}{2d+1} \frac{(-1)^m}{2m-2d} \bernoulli{2m-2d} = 0
\end{equation*}
i.e
\begin{equation*}
    \coeffA{m}{d} = (-1)^{m-1} \frac{(2m+1)!}{d!d!m!(m-2d-1)!} \frac{1}{m-d} \bernoulli{2m-2d}
\end{equation*}
Continue similarly we can compute $\coeffA{m}{r}$ for each integer $r$ in range $m/2^{s+1}\leq r < m/2^s$
(iterating consecutively $s=1,2,\ldots$) via previously determined values of $\coeffA{m}{d}$ as follows
\begin{equation*}
    \coeffA{m}{r} =
    (2r+1) \binom{2r}{r} \sum_{d \geq 2r+1}^{m} \coeffA{m}{d} \binom{d}{2r+1} \frac{(-1)^{d-1}}{d-r}
    \bernoulli{2d-2r}
\end{equation*}
Finally, we are capable to define the coefficient $\coeffA{m}{r}$ via the next recurrent relation
\begin{defn} (Definition of the real coefficients $\coeffA{m}{r}$.)
    \begin{equation}
        \label{eq:definition_coefficient_a}
        \coeffA{m}{r} \colonequals
        \begin{cases}
        (2r+1)
            \binom{2r}{r} & \mathrm{if} \; r=m \\
            (2r+1) \binom{2r}{r} \sum_{d \geq 2r+1}^{m} \coeffA{m}{d} \binom{d}{2r+1} \frac{(-1)^{d-1}}{d-r}
            \bernoulli{2d-2r} & \mathrm{if} \; 0 \leq r<m \\
            0 & \mathrm{if} \; r<0 \; \mathrm{or} \; r>m
        \end{cases}
    \end{equation}
\end{defn}
where $\bernoulli{t}$ are Bernoulli numbers~\cite{bateman1953higher}.
It is assumed that $\bernoulli{1}=\frac{1}{2}$.
For example,
\begin{table}[H]
    \begin{center}
        \setlength\extrarowheight{-6pt}
        \begin{tabular}{c|cccccccc}
            $m/r$ & 0 & 1       & 2      & 3      & 4   & 5    & 6     & 7 \\ [3px]
            \hline
            0     & 1 &         &        &        &     &      &       &       \\
            1     & 1 & 6       &        &        &     &      &       &       \\
            2     & 1 & 0       & 30     &        &     &      &       &       \\
            3     & 1 & -14     & 0      & 140    &     &      &       &       \\
            4     & 1 & -120    & 0      & 0      & 630 &      &       &       \\
            5     & 1 & -1386   & 660    & 0      & 0   & 2772 &       &       \\
            6     & 1 & -21840  & 18018  & 0      & 0   & 0    & 12012 &       \\
            7     & 1 & -450054 & 491400 & -60060 & 0   & 0    & 0     & 51480
        \end{tabular}
    \end{center}
    \caption{Coefficients $\coeffA{m}{r}$.}
    \label{tab:table_of_coefficients_a}
\end{table}
The nominators and denominators of the coefficients $\coeffA{m}{r}$ are also registered as sequences
in OEIS~\cite{kolosov2018numerator,kolosov2018denominator}.
It is as well interesting to notice that row sums of the $\coeffA{m}{r}$ give powers of $2$
\begin{equation*}
    \sum_{r=0}^{m} \coeffA{m}{r} = 2^{2m+1} - 1
\end{equation*}
Let be a theorem
\begin{thm}
    For every $n\geq 1, \; n,m\in\mathbb{N}$ there are $\coeffA{m}{0}, \coeffA{m}{1},\dots,\coeffA{m}{m}$,
    such that
    \begin{equation*}
        n^{2m+1} = \sum_{k=1}^{n}\sum_{r=0}^{m} \coeffA{m}{r} k^r (n-k)^r
        \label{eq:odd-power-theorem}
    \end{equation*}
    where $\coeffA{m}{r}$ is a real coefficient defined recursively by~\eqref{eq:definition_coefficient_a}.
\end{thm}
Finally, we got our road to the main definition of the polynomial $\polynomialP{m}{b}{x}$.
Introducing the parameter $b$ to the upper summation bound of the equation~\eqref{eq:odd-power-theorem},
we have the definition
\begin{defn} (Polynomial $\polynomialP{m}{b}{x} $of degree $2m+1$.)
    \begin{equation}
        \polynomialP{m}{b}{x} = \sum_{k=0}^{b-1} \sum_{r=0}^{m} \coeffA{m}{r} k^r(x-k)^r
        \label{eq:definition_polynomial_p}
    \end{equation}
\end{defn}
where $\coeffA{m}{r}$ are real coefficients~\eqref{eq:definition_coefficient_a}.
A comprehensive discussion on the polynomial $\polynomialP{m}{b}{x}$ as well as its properties can be found at~\cite{kolosov2016link}.
In 2023, Albert Tkaczyk yet again extended the theorem~\eqref{eq:odd-power-theorem} to the so-called three dimension case
so that it gives polynomials of the form $n^{3l+2}$ at~\cite{albert_tkaczyk_2023_8371454}.
