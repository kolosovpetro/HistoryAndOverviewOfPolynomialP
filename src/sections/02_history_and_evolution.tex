Back than, in 2016 I remember myself playing with finite differences of polynomial $n^3$ over the domain of natural
numbers $n\in\mathbb{N}$ and first very naive question that came to my mind was \textit{Is it possible to assemble the value of
    $n^3$ backwards having finite differences?}
Definitely, answer to this question is \textit{Yes} and has been well-developed in between 1674--1684
by Issac Newton's fundamental work on the topic that is nowadays known as foundation of classical interpolation theory.
That time, in 2016, due to lack of knowledge and perspective of view I started re-inventing the interpolation formula by myself,
fueled by purest interest and sense of mystery.
All mathematical laws and relations exist till the very beginning, but we only find and describe them,
I was inspired by that mindset and started my own journey.
So let's begin considering the table of finite differences of the polynomial $n^3$
\begin{table}[H]
    \begin{center}
        \setlength\extrarowheight{-6pt}
        \begin{tabular}{c|cccc}
            $n$ & $n^3$ & $\Delta(n^3)$ & $\Delta^2(n^3)$ & $\Delta^3(n^3)$ \\
            \hline
            0   & 0     & 1             & 6               & 6               \\
            1   & 1     & 7             & 12              & 6               \\
            2   & 8     & 19            & 18              & 6               \\
            3   & 27    & 37            & 24              & 6               \\
            4   & 64    & 61            & 30              & 6               \\
            5   & 125   & 91            & 36              &                 \\
            6   & 216   & 127           &                 &                 \\
            7   & 343   &               &                 &
        \end{tabular}
    \end{center}
    \caption{Table of finite differences of the polynomial $n^3$.} \label{tab:table}
\end{table}
First and foremost, we can observe that finite difference $\Delta(n^3)$ of the polynomial $n^3$
can be expressed via summation over $n$, e.g
\begin{align}
    \label{eq:cubes_interpolation}
    \begin{split}
        \Delta(0^3) &= 1+6 \cdot 0 \\
        \Delta(1^3) &= 1+6\cdot0+6\cdot1 \\
        \Delta(2^3) &= 1+6\cdot0+6\cdot1+6\cdot2 \\
        \Delta(3^3) &= 1+6\cdot0+6\cdot1+6\cdot2+6\cdot3 \\
        &\; \; \vdots \\
        \Delta(n^3) &= 1+6\cdot0+6\cdot1+6\cdot2+6\cdot3+\cdots+6\cdot n = 1 + 6 \sum_{k=0}^{n} k
    \end{split}
\end{align}
The one experienced mathematician would immediately notice a spot to apply Faulhaber's formula to expand the term
$\sum_{k=0}^{n} k$ reaching expected result that matches Binomial theorem, so that
\begin{equation*}
    \sum_{k=0}^{n} k = \frac{1}{2}(n+n^2)
\end{equation*}
Then ours above relation $\Delta(n^3) = 1+6\cdot0+6\cdot1+6\cdot2+6\cdot3+\cdots+6\cdot n = 1 + 6 \sum_{k=0}^{n} k$
immediately terms into Binomial expansion
\begin{equation}
    \Delta(n^3) = (n+1)^3 - n^3 = 1 + 6 \left[ \frac{1}{2}(n+n^2) \right] = 1 + 3 n + 3 n^2 = \sum_{k=0}^{2} \binom{3}{k} n^k
    \label{eq:cubes-difference-binomial-theorem}
\end{equation}
However, as it said, I was not experienced mathematician back than, so that it terned out for me a little different perspective.
Not following the convenient solution~\eqref{eq:cubes-difference-binomial-theorem},
I rearranged the finite differences from table~\eqref{tab:table} explicitly to get
\begin{align*}
    n^3 &= [1+6\cdot0]+[1+6\cdot0+6\cdot1]+[1+6\cdot0+6\cdot1+6\cdot2]+\cdots \\
    &+[1+6\cdot0+6\cdot1+6\cdot2+\cdots+6\cdot(n-1)]
\end{align*}
And then combined under the summation in terms of $(n-k)$
\[
    n^3 = n +(n-0) \cdot6 \cdot0 + (n-1)\cdot6\cdot1 + (n-2)\cdot6\cdot2 + \cdots+1\cdot6\cdot(n-1)
\]
Therefore, the polynomial $n^3$ can be considered as follows
\begin{equation}
    \label{eq:cube_identity}
    n^3 = \sum_{k=1}^{n} 6k(n-k) + 1
\end{equation}
