\subsection{System of linear equations approach}\label{subsec:system-of-linear-equations-approach}
Soon enough the question~\eqref{question:higher_powers} got attention from other people.
In 2018, Albert Tkaczyk has published two of his works~\cite{tkaczyk2018problem, tkaczyk2018continuation}
showing the cases for polynomials $n^5, \; n^7$ and $n^9$ that were obtained similarly as~\eqref{eq:cube_identity}.
In short, it appears that relation~\eqref{eq:cube_identity} could be generalized
for any non-negative odd power $2m+1$ solving a system of linear equations.
It was proposed that the case for $n^5$ has explicit form
\begin{equation*}
    n^5 = \sum_{k=1}^{n} \left[ A k^2(n-k)^2 + Bk(n-k) + C \right]
\end{equation*}
where $A,B,C$ are yet-unknown coefficients.
Denote $A,B,C$ as $\coeffA{2}{0}, \coeffA{2}{1}, \coeffA{2}{2}$
to reach the form of a compact double sum
\begin{equation*}
    n^5 = \sum_{k=1}^{n} \sum_{r=0}^{2} \coeffA{2}{r} k^r (n-k)^r
\end{equation*}
Observing the equation above, the potential form of generalized odd-power identity becomes more obvious.
To evaluate the coefficients $\coeffA{2}{0}, \coeffA{2}{1}, \coeffA{2}{2}$
it is necessary construct and solve a system of linear equations following the process
\begin{equation*}
    \begin{split}
        n^5 &= \sum_{r=0}^{2} \coeffA{2}{r} \sum_{k=1}^{n} k^r (n-k)^r \\
        &= \coeffA{2}{0} \sum_{k=1}^{n} k^0 (n-k)^0 + \coeffA{2}{1} \sum_{k=1}^{n} k^1 (n-k)^1 + \coeffA{2}{2} \sum_{k=1}^{n} k^2 (n-k)^2
    \end{split}
\end{equation*}
Expand the terms $\sum_{k=1}^{n} k^r (n-k)^r$ applying the
Faulhaber's formula~\cite{beardon1996sums}
to get the equation
\begin{equation*}
    \coeffA{m}{0} n
    + \coeffA{m}{1} \left[ \frac{1}{6} (-n + n^3) \right]
    + \coeffA{m}{2} \left[ \frac{1}{30} (-n + n^5) \right] - n^5 = 0
\end{equation*}
Multiplying by $30$ both right-hand side and left-hand side, we get
\begin{equation*}
    30 \coeffA{2}{0} n + 5 \coeffA{2}{1} (-n + n^3) + \coeffA{2}{2} (-n + n^5) - 30n^5 = 0
\end{equation*}
Expanding the brackets and rearranging the terms gives
\begin{equation*}
    30 \coeffA{2}{0} - 5 \coeffA{2}{1} n + 5 \coeffA{2}{1} n^3 - \coeffA{2}{2} n + \coeffA{2}{2} n^5 - 30n^5 = 0
\end{equation*}
Combining the common terms yields
\begin{equation*}
    n (30 \coeffA{2}{0} - 5 \coeffA{2}{1} - \coeffA{2}{2}) + 5 \coeffA{2}{1} n^3 + n^5 (\coeffA{2}{2} - 30) = 0
\end{equation*}
Therefore, the system of linear equations follows
\begin{equation*}
    \begin{cases}
        30 \coeffA{2}{0} - 5 \coeffA{2}{1} - \coeffA{2}{2} = 0 \\
        \coeffA{2}{1} = 0 \\
        \coeffA{2}{2} - 30 = 0
    \end{cases}
\end{equation*}
Solving it, we get
\begin{equation*}
    \begin{cases}
        \coeffA{2}{2} = 30 \\
        \coeffA{2}{1} = 0 \\
        \coeffA{2}{0} = 1
    \end{cases}
\end{equation*}
So that the odd-power identity holds
\begin{equation*}
    n^5 = \sum_{k=1}^{n} 30k^2(n-k)^2 + 1
\end{equation*}
It is also clearly seen
why the above identity is true by arranging the terms $30k^2(n-k)^2 + 1$ over $0 \leq k \leq n$ as tabular.
See the OEIS sequence~\cite{oeis_numerical_triangle_row_sums_give_fifth_powers}
\begin{table}[H]
    \setlength\extrarowheight{-6pt}
    \begin{tabular}{c|cccccccc}
        $n/k$ & 0 & 1    & 2    & 3    & 4    & 5    & 6    & 7 \\
        \hline
        0     & 1 &      &      &      &      &      &      &   \\
        1     & 1 & 1    &      &      &      &      &      &   \\
        2     & 1 & 31   & 1    &      &      &      &      &   \\
        3     & 1 & 121  & 121  & 1    &      &      &      &   \\
        4     & 1 & 271  & 481  & 271  & 1    &      &      &   \\
        5     & 1 & 481  & 1081 & 1081 & 481  & 1    &      &   \\
        6     & 1 & 751  & 1921 & 2431 & 1921 & 751  & 1    &   \\
        7     & 1 & 1081 & 3001 & 4321 & 4321 & 3001 & 1081 & 1
    \end{tabular}
    \caption{Values of $30k^2(n-k)^2 + 1$.
    See the OEIS entry \href{https://oeis.org/A300656}{\texttt{A300656}}.}
    \label{tab:row-sums-gives-fifth-power}
\end{table}


Now we can see that the relation~\eqref{eq:cube_identity}
we got via interpolation of cubes
can be generalized for all non-negative odd-powers $2m+1$ by constructing
and solving a certain system of linear equations.
Therefore, the generalized form of odd-power identity has the form
\begin{equation}
    n^{2m+1} = \sum_{r=0}^{m} \coeffA{m}{r} \sum_{k=1}^{n} k^{r} (n-k)^{r}\label{eq:odd-power-identity}
\end{equation}
where $\coeffA{m}{r}$ are real coefficients.
In more details, the identity~\eqref{eq:odd-power-identity} is discussed
separately in~\cite{unusual_identity_for_odd_powers, polynomial_identity_with_binomial_theorem_and_faulhabers_formula}.

As one more example, let be $m=3$ so that we have the following relation defined by~\eqref{eq:odd-power-identity}
\begin{equation*}
    \coeffA{m}{0} n
    + \coeffA{m}{1} \left[ \frac{1}{6} (-n + n^3) \right]
    + \coeffA{m}{2} \left[ \frac{1}{30} (-n + n^5) \right]
    + \coeffA{m}{3} \left[ \frac{1}{420} (-10 n + 7 n^3 + 3 n^7) \right] - n^7 = 0
\end{equation*}
Multiplying by $420$ right-hand side and left-hand side, we get
\begin{equation*}
    420 \coeffA{3}{0} n + 70 \coeffA{2}{1} (-n + n^3) + 14 \coeffA{2}{2} (-n + n^5) + \coeffA{3}{3} (-10 n + 7 n^3 + 3 n^7) - 420n^7 = 0
\end{equation*}
Expanding brackets and rearranging the terms gives
\begin{equation*}
    \begin{split}
        420 \coeffA{3}{0} n
        &- 70 \coeffA{3}{1} + 70 \coeffA{3}{1} n^3 - 14 \coeffA{3}{2} n + 14 \coeffA{3}{2} n^5 \\
        &- 10 \coeffA{3}{3} n + 7 \coeffA{3}{3} n^3 + 3 \coeffA{3}{3} n^7 - 420n^7 = 0
    \end{split}
\end{equation*}
Combining the common terms yields
\begin{equation*}
    \begin{split}
        &n (420 \coeffA{3}{0} - 70 \coeffA{3}{1} - 14 \coeffA{3}{2} - 10 \coeffA{3}{3}) \\
        &+ n^3 (70 \coeffA{3}{1} + 7 \coeffA{3}{3})
        + n^5 14 \coeffA{3}{2}
        + n^7 (3 \coeffA{3}{3} - 420)
        = 0
    \end{split}
\end{equation*}
Therefore, the system of linear equations follows
\begin{equation*}
    \begin{cases}
        420 \coeffA{3}{0} - 70 \coeffA{3}{1} - 14 \coeffA{3}{2} - 10 \coeffA{3}{3} = 0 \\
        70 \coeffA{3}{1} + 7 \coeffA{3}{3} = 0 \\
        \coeffA{3}{2} - 30 = 0 \\
        3 \coeffA{3}{3} - 420 = 0
    \end{cases}
\end{equation*}
Solving it, we get
\begin{equation*}
    \begin{cases}
        \coeffA{3}{3} = 140 \\
        \coeffA{3}{2} = 0 \\
        \coeffA{3}{1} = -\frac{7}{70} \coeffA{3}{3} = -14 \\
        \coeffA{3}{0} = \frac{(70 \coeffA{3}{1} + 10 \coeffA{3}{3})}{420} = 1
    \end{cases}
\end{equation*}
So that odd-power identity~\eqref{eq:odd-power-identity} holds
\begin{equation*}
    n^7 = \sum_{k=1}^{n} 140 k^3 (n-k)^3 - 14k(n-k) + 1
\end{equation*}
It is also clearly seen
why the above identity is true evaluating the terms $140 k^3 (n-k)^3 - 14k(n-k) + 1$ over $0 \leq k \leq n$ as
the OEIS sequence \href{https://oeis.org/A300785}{\texttt{A300785}}~\cite{oeis_numerical_triangle_row_sums_give_seventh_powers} shows.
\begin{table}[H]
    \setlength\extrarowheight{-6pt}
    \begin{tabular}{c|cccccccc}
        $n/k$ & 0 & 1     & 2      & 3      & 4      & 5      & 6     & 7 \\
        \hline
        0     & 1 &       &        &        &        &        &       &   \\
        1     & 1 & 1     &        &        &        &        &       &   \\
        2     & 1 & 127   & 1      &        &        &        &       &   \\
        3     & 1 & 1093  & 1093   & 1      &        &        &       &   \\
        4     & 1 & 3739  & 8905   & 3739   & 1      &        &       &   \\
        5     & 1 & 8905  & 30157  & 30157  & 8905   & 1      &       &   \\
        6     & 1 & 17431 & 71569  & 101935 & 71569  & 17431  & 1     &   \\
        7     & 1 & 30157 & 139861 & 241753 & 241753 & 139861 & 30157 & 1
    \end{tabular}
    \caption{Values of $140 k^3 (n-k)^3 - 14k(n-k) + 1$.
    See the OEIS entry \href{https://oeis.org/A300785}{\texttt{A300785}}
    ~\cite{oeis_numerical_triangle_row_sums_give_seventh_powers}.}
    \label{tab:row-sums-gives-seventh-power}
\end{table}

