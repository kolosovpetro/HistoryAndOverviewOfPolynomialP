In this section let's give a short overview of related works that are based onto definition
of polynomials $\polynomialP{m}{b}{x}$.
In~\cite{kolosov2023another} is given a relation in terms of partial differential differential equations such that
ordinary derivative of odd-power $2m+1$ can be reached in terms of partial derivatives of $\polynomialP{m}{b}{x}$.
Let be a fixed point $v\in \mathbb{N}$, then ordinary derivative $\frac{d}{dx} g_v (u)$ of the odd-power function $g_v(x) = x^{2v + 1}$
evaluate in point $u\in\mathbb{R}$ equals to partial derivative $(f_{v})^{'}_{x} (u, u)$ evaluate in point $(u, u)$ plus
partial derivative $(f_{v})^{'}_{z} (u, u)$ evaluate in point $(u, u)$
\begin{equation}
    \frac{d}{dx} g_v (u) = (f_{v})^{'}_{x} (u, u) + (f_{v})^{'}_{z} (u, u)
    \label{eq:odd-exponential-identity}
\end{equation}
where $f_{y} (x, z) = \sum_{k=1}^{z} \sum_{r=0}^{y} \coeffA{y}{r} k^r (x-k)^r = \polynomialP{y}{z}{x}$.
Afterwards, the equation~\eqref{eq:odd-exponential-identity}
is generalized over the timescales $\mathbb{T} \times \mathbb{T}$ providing its dynamic equation analog
in~\cite{kolosov2016study}.

Second article~\cite{kolosov_2024_10575485} gives another perspective of ordinary derivatives of polynomials expressing
them via double limit as
\[
    \lim_{h \to 0} \polynomialP{m}{x+h}{x} = x^{2m+1}
\]
that opens such opportunity.

In~\cite{barbero2020two} based on~\eqref{eq:equation7}, the authors give a new identity involving
Bernoulli polynomials and combinatorial numbers that provides,
in particular, a Faulhaber-like formula for sums of the form $1^m(n-1)^m + 2^m (n -2)^m + \cdots + (n - 1)^m 1^m$ for
positive integers $m$ and $n$.

Few sequences were contributed to the
OEIS~\cite{kolosov2018coefficientspolynomial1, kolosov2018coefficientspolynomial2, kolosov2018coefficientspolynomial3}
showing the coefficients of the polynomial $\polynomialP{m}{b}{x}$ having fixed points $m,b$ while $x\in\mathbb{R}$.
